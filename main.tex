\documentclass[a4paper,12pt]{article}
\usepackage[english]{babel}
\usepackage[utf8]{inputenc}

%
% For alternative styles, see the biblatex manual:
% http://mirrors.ctan.org/macros/latex/contrib/biblatex/doc/biblatex.pdf
%
% The 'verbose' family of styles produces full citations in footnotes, 
% with and a variety of options for ibidem abbreviations.
%
\usepackage{csquotes}
\usepackage[style=verbose-ibid,backend=bibtex]{biblatex}
\bibliography{sample}

\usepackage{lipsum} % for dummy text

\title{Elaboration 1}

\author{John Hundley}

\date{\today}

\begin{document}
\maketitle

\section{Elaboration, Tools, and Revised Method:}

What will be principally tested is the feasibility of using a search engine to procure multiple documents for key terms or themes. This was identified as a major problem in my research in my Scoping Exercise last week. Ideally, the aim is to perform a single search that might achieve this. The most appropriate software at this stage appears to be Voyant. Whilst this program appears not to perform a thematic analysis, it can produce word frequency analysis and frequency distribution graphs among a series of documents or databases. However, there might be a way to test a 'thematic' search in the broadest possible terms. If it cannot perform a basic thematic search then another program will have to be used instead. One benefit of this program is that it is an open-source project. It is also available on Git-Hub. The risks are that the difficulty of using the code of the program are unknown. If it is difficult to use then this presents several problems, namely available time and lack of skills. Additionally, there are several other programs that also need to be tested. Firstly, Zotero needs be tested to assure it can generate an appropriate bibliography. Secondly, toggl also needs to be assessed to see whether it is needed during the research process. Thirdly, Cloudstor ought to be tested to see if it can store files safely and securely. Finally, the databases identified below should be tested to confirm whether special access, such as a subscription, is needed or not. The method as identified last week, has been revised to include the following: 


\begin{itemize} 

\item 1.	Identify thesis or argument. 
\item 2.	Identify relevant sources. 
\item 3.	Search necessary data bases. This includes Google Scholar, Macquarie University Library, NewCat (Newcastle University Database), JSTOR, and Trove. 
\item 4.	Access and download required sources. This would require access to aforementioned databases. Sometimes this requires a university enrolment ID or a subscription service. 
\item 5.	Use Zotero toolbar when accessing files to record and manage bibliography.
\item 6.	Store sources in files kept on Couldstor and OneDrive. 
\item 7.	Open Voyant and upload sources in search engine. Perform search. 
\item 8.	Identify key themes, terms, and connections as demonstrated in Voyant. This might be indicated in the tables, graphs, and charts that Voyant provides. 
\item 9.	Collate this information and any metadata in a Word document or Cloudstor file. If necessary, record time stamps on toggl (time stamp tool). Save work on Cloudstor. 
\item 10.	Read documents on either a PDF view or printed.
\item 11.	Write response on Word or Cloudstor. Work to be saved to computer hard-drive and Cloudstor folder. 
\item 12.	Use Zotero to generate bibliography. 

\end{document}